\documentclass[handout]{beamer}
%\documentclass[handout]{beamer}

\usepackage{graphicx}

\usepackage{amsmath}
\usepackage{xcolor}
\usepackage{color}
\usepackage{listings}
\usepackage{etoolbox}
\usepackage{calc}

\setlength{\fboxsep}{0.5mm}

\definecolor{orange}{rgb}{1,0.5,0}
\definecolor{green}{rgb}{0,0.5,0}
\definecolor{gray}{rgb}{0.25,0.25,0.25}
\definecolor{mauve}{rgb}{0.58,0,0.82}
\definecolor{red}{rgb}{1.0,0,0}

\colorlet{punct}{green}
\colorlet{delim}{green}
\colorlet{num}{mauve}

\definecolor{codebg}{rgb}{1.0,1.0,0.8}
\definecolor{jargonbg}{rgb}{0.8,1.0,0.8}
\definecolor{spacebg}{rgb}{1.0,0.8,0.0}
\definecolor{lightspacebg}{rgb}{1.0,0.9,0.5}

\newtoggle{InString}{}% Keep track of if we are within a string
\togglefalse{InString}% Assume not initally in string

\newcommand*{\ColorIfNotInString}[1]{\iftoggle{InString}{#1}{\color{num}#1}}
\newcommand*{\ProcessQuote}[1]{\color{red}#1\iftoggle{InString}{\global\togglefalse{InString}}{\global\toggletrue{InString}}}

\lstdefinelanguage{py}{
	morekeywords={if,elif,else,for,in,while,break,continue,def,class,True,False,print,return,and,or,not,with,None,try,except,finally,raise,as,input,open,len},
	morecomment=[l]\#,
	literate=
	*{=}{{{\color{punct}{=}}}}{1}
	{|}{{{\color{punct}{|}}}}{1}
	{\^}{{{\color{punct}{\^{}}}}}{1}
	{\{}{{{\color{punct}{\{}}}}{1}
	{\}}{{{\color{punct}{\}}}}}{1}
	{[}{{{\color{punct}{[}}}}{1}
	{]}{{{\color{punct}{]}}}}{1}
	{)}{{{\color{punct}{)}}}}{1}
	{(}{{{\color{punct}{(}}}}{1}
	{<}{{{\color{punct}{<}}}}{1}
	{>}{{{\color{punct}{>}}}}{1}
	{*}{{{\color{punct}{*}}}}{1}
	{?}{{{\color{punct}{?}}}}{1}
	{:}{{{\color{punct}{:}}}}{1}
	{,}{{{\color{punct}{,}}}}{1}
	{\&}{{{\color{punct}{\&}}}}{1}
	{!}{{{\color{punct}{!}}}}{1}
	{-}{{{\color{punct}{-}}}}{1}
	{+}{{{\color{punct}{+}}}}{1}
	{.}{{{\color{punct}{.}}}}{1}
	{\%}{{{\color{punct}{\%}}}}{1}
	{/}{{{\color{punct}{/}}}}{1}
	{\~}{{{$\color{punct}{\sim}$}}}{1}
	{;}{{{\color{punct}{;}}}}{1}
	%{"}{{{\ProcessQuote{"}}}}1% Disable coloring within double quotes
	%{'}{{{\ProcessQuote{'}}}}1% Disable coloring within single quote
	{0}{{{\ColorIfNotInString{0}}}}1
	{1}{{{\ColorIfNotInString{1}}}}1
	{2}{{{\ColorIfNotInString{2}}}}1
	{3}{{{\ColorIfNotInString{3}}}}1
	{4}{{{\ColorIfNotInString{4}}}}1
	{5}{{{\ColorIfNotInString{5}}}}1
	{6}{{{\ColorIfNotInString{6}}}}1
	{7}{{{\ColorIfNotInString{7}}}}1
	{8}{{{\ColorIfNotInString{8}}}}1
	{9}{{{\ColorIfNotInString{9}}}}1,
}

\lstset{
	language=py,
	%
	captionpos=b,                    % sets the caption-position to bottom
	mathescape=true,
	keepspaces=true,
	showspaces=false,
	showstringspaces=false,          % underline spaces within strings only
	showtabs=false,                  % show tabs within strings adding particular underscores
	stepnumber=1,
	tabsize=4,
	title=\lstname,
	morestring=[b]",
	morestring=[b]',
	belowskip=-0.8 \baselineskip,
	%
	numberstyle=\footnotesize\color{gray},
	rulecolor=\color{black},
	basicstyle=\ttfamily\small,
	backgroundcolor=\color{codebg},
	keywordstyle=\color{blue},
	commentstyle=\color{orange},
	stringstyle=\color{red},
}

\lstdefinestyle{output}{
	basicstyle=\small\ttfamily,
	numbers=none,
	frame=tblr,
	columns=fullflexible,
	backgroundcolor=\color{blue!10},
	linewidth=0.9\linewidth,
	xleftmargin=0.1\linewidth
}

\newlength{\DepthReference}
\settodepth{\DepthReference}{g}
\newlength{\HeightReference}
\settoheight{\HeightReference}{T}
\newlength{\Width}
\newcommand{\evenbox}[2]{\colorbox{#1}{\rule[-2pt]{0pt}{10pt}#2}}

\title{CUCaTS Python Workshop}
\date{Lent 2015}

\begin{document}
	\begin{frame}[fragile]
		\frametitle{Making change}
		\footnotesize
		Cast your mind back to Friday night. The scene is the ADC bar. It's late, you've had a few.
		
		But not enough! So you wander over to the barman and mutter something about a Staropramen. ``2.65 please".
		
		You reach for your wallet, but crap! The ADC doesn't take cards!
		
		You rustle through your pockets and pull out the contents. No notes, but a whole load of coins. 5ps, 10ps, 20ps. You look at it and think to yourself ``I'm... sure this is enough to make 2.65, but the question is... how?"
		
		That is the question we will answer today, and we will answer it with Python!
	\end{frame}

	\begin{frame}[fragile]
		\frametitle{Problem data}
		
		Going back to the first session: we told you that programming was about processing data. In this problem, what data are we going to be dealing with?
		
		\pause
		
		Input:
		\begin{itemize}
			\item \lstinline|amount| \\
			      How much we want to make
			\item \lstinline|coins| \\
			      The coins that we have to make it
		\end{itemize}
		
		\pause
		
		Output:
		\begin{itemize}
			\item \lstinline|ways| \\
			      All the ways of making the amount
				\begin{lstlisting}
				[[100, 5, 20, 20, 20, 20, 20, 20, 20, 20],
				 [200, 50, 10, 5]]
				\end{lstlisting}
		\end{itemize}
	\end{frame}
		
	\begin{frame}[fragile]
		\frametitle{Problem data - Types}
		
		Let's be more specific about our data. Input:
		
		\begin{itemize}
			\item The amount will simply be a number, either integer or float. For simplicity, we'll say its the number of pennies, so an integer.
			\item \lstinline|amount|: integer \\
			\item The coins we have will then just be a list of integers.
			\item \lstinline|coins|: list of integer \\
		\end{itemize}
		
		Output:
		\begin{itemize}
			\item  If we were working out just a \emph{single} way of making the amount, it would be a list of integers, that is, all the coins needed. But we're working out all the possible ways, so our result will be a list of list of integers: each element of the list will itself be another way of making the amount, a list of integers.
			\item \lstinline|ways|: list of list of integers \\
		\end{itemize}
		
	\end{frame}
	
	
	\begin{frame}[fragile]
		\frametitle{}
		
		Writing that up as Python:
		
		\begin{lstlisting}
		def makechange(amount, coins):
		    ways = []
		
		    $\textnormal{[\dots work out all the ways\dots]}$
		
		    return ways
		\end{lstlisting}
		
		Start by filtering out all the coins that are too large to make the amount from our list of \lstinline|coins|. The (an) answer is on the next slide, but try it yourself before looking!
	\end{frame}
	
	\begin{frame}[fragile]
		\frametitle{}
		
		I've used a list comprehension to reproduce the list of coins, but leaving only ones that can fit in \lstinline|amount|.
		
		\begin{lstlisting}[basicstyle=\tt\footnotesize]
		def makechange(amount, coins):
		    ways = []
		
		    #Get rid of the coins that are too large
		    coins = [coin for coin in coins if coin <= amount]
		
		    $\textnormal{[\dots]}$
		
		    return ways
		\end{lstlisting}
	\end{frame}
	
	\begin{frame}[fragile]
		\frametitle{}
		
		Now we'll go through our list of coins in turn\dots
		
		\begin{lstlisting}
		def makechange(amount, coins):
		    ways = []
		
		    #Get rid of the coins that are too large
		    coins = [coin for coin in coins if coin <= amount]
		
		    for coin in coins:
		        $\textnormal{[\dots]}$
		
		    return ways
		\end{lstlisting}
	\end{frame}
	
	\begin{frame}[fragile]
		\frametitle{}
		
		\begin{lstlisting}
		def makechange(amount, coins):
		    #Deal with the trivial case of making zero
		    if amount == 0:
		        return [[]]
		
		    else:
		        ways = []
		
		    #Get rid of the coins that are too large
		    coins = [coin for coin in coins if coin <= amount]
		
		    for coin in coins:
		        ways += also(coin, makechange(amount-coin, coins))
		
		        #Exclude this coin from any more solutions
		        coins = coins[1:]
		
		    return ways
		\end{lstlisting}
	\end{frame}
	
\end{document}