\documentclass[notes]{beamer}
%\documentclass[handout]{beamer}

\usepackage{graphicx}

\usepackage{amsmath}
\usepackage{xcolor}
\usepackage{color}
\usepackage{listings}
\usepackage{etoolbox}
\usepackage{calc}

\setlength{\fboxsep}{0.5mm}

\definecolor{orange}{rgb}{1,0.5,0}
\definecolor{green}{rgb}{0,0.5,0}
\definecolor{gray}{rgb}{0.25,0.25,0.25}
\definecolor{mauve}{rgb}{0.58,0,0.82}
\definecolor{red}{rgb}{1.0,0,0}

\colorlet{punct}{green}
\colorlet{delim}{green}
\colorlet{num}{mauve}

\definecolor{codebg}{rgb}{1.0,1.0,0.8}
\definecolor{jargonbg}{rgb}{0.8,1.0,0.8}
\definecolor{spacebg}{rgb}{1.0,0.8,0.0}
\definecolor{lightspacebg}{rgb}{1.0,0.9,0.5}

\newtoggle{InString}{}% Keep track of if we are within a string
\togglefalse{InString}% Assume not initally in string

\newcommand*{\ColorIfNotInString}[1]{\iftoggle{InString}{#1}{\color{num}#1}}
\newcommand*{\ProcessQuote}[1]{\color{red}#1\iftoggle{InString}{\global\togglefalse{InString}}{\global\toggletrue{InString}}}

\lstdefinelanguage{py}{
	morekeywords={if,elif,else,for,in,while,break,continue,def,class,True,False,print,return,and,or,not,with,None,try,except,finally,raise,as,input,open,len},
	morecomment=[l]\#,
	literate=
	*{=}{{{\color{punct}{=}}}}{1}
	{|}{{{\color{punct}{|}}}}{1}
	{\^}{{{\color{punct}{\^{}}}}}{1}
	{\{}{{{\color{punct}{\{}}}}{1}
	{\}}{{{\color{punct}{\}}}}}{1}
	{[}{{{\color{punct}{[}}}}{1}
	{]}{{{\color{punct}{]}}}}{1}
	{)}{{{\color{punct}{)}}}}{1}
	{(}{{{\color{punct}{(}}}}{1}
	{<}{{{\color{punct}{<}}}}{1}
	{>}{{{\color{punct}{>}}}}{1}
	{*}{{{\color{punct}{*}}}}{1}
	{?}{{{\color{punct}{?}}}}{1}
	{:}{{{\color{punct}{:}}}}{1}
	{,}{{{\color{punct}{,}}}}{1}
	{\&}{{{\color{punct}{\&}}}}{1}
	{!}{{{\color{punct}{!}}}}{1}
	{-}{{{\color{punct}{-}}}}{1}
	{+}{{{\color{punct}{+}}}}{1}
	{.}{{{\color{punct}{.}}}}{1}
	{\%}{{{\color{punct}{\%}}}}{1}
	{/}{{{\color{punct}{/}}}}{1}
	{\~}{{{$\color{punct}{\sim}$}}}{1}
	{;}{{{\color{punct}{;}}}}{1}
	%{"}{{{\ProcessQuote{"}}}}1% Disable coloring within double quotes
	%{'}{{{\ProcessQuote{'}}}}1% Disable coloring within single quote
	{0}{{{\ColorIfNotInString{0}}}}1
	{1}{{{\ColorIfNotInString{1}}}}1
	{2}{{{\ColorIfNotInString{2}}}}1
	{3}{{{\ColorIfNotInString{3}}}}1
	{4}{{{\ColorIfNotInString{4}}}}1
	{5}{{{\ColorIfNotInString{5}}}}1
	{6}{{{\ColorIfNotInString{6}}}}1
	{7}{{{\ColorIfNotInString{7}}}}1
	{8}{{{\ColorIfNotInString{8}}}}1
	{9}{{{\ColorIfNotInString{9}}}}1,
}

\lstset{
	language=py,
	%
	captionpos=b,                    % sets the caption-position to bottom
	mathescape=true,
	keepspaces=true,
	showspaces=false,
	showstringspaces=false,          % underline spaces within strings only
	showtabs=false,                  % show tabs within strings adding particular underscores
	stepnumber=1,
	tabsize=4,
	title=\lstname,
	morestring=[b]",
	morestring=[b]',
	belowskip=-0.8 \baselineskip,
	%
	numberstyle=\footnotesize\color{gray},
	rulecolor=\color{black},
	basicstyle=\ttfamily\small,
	backgroundcolor=\color{codebg},
	keywordstyle=\color{blue},
	commentstyle=\color{orange},
	stringstyle=\color{red},
}

\lstdefinestyle{output}{
	basicstyle=\small\ttfamily,
	numbers=none,
	frame=tblr,
	columns=fullflexible,
	backgroundcolor=\color{blue!10},
	linewidth=0.9\linewidth,
	xleftmargin=0.1\linewidth
}

\newlength{\DepthReference}
\settodepth{\DepthReference}{g}
\newlength{\HeightReference}
\settoheight{\HeightReference}{T}
\newlength{\Width}
\newcommand{\evenbox}[2]{\colorbox{#1}{\rule[-2pt]{0pt}{10pt}#2}}

\title{CUCaTS Python Workshop - Session 2}
\date{Lent 2015}

\begin{document}
	\begin{frame}[fragile]
		\maketitle
	\end{frame}
	
	\begin{frame}[fragile]
		\frametitle{Values and Variables}
		Remember:
		\begin{itemize}
			\item Values are pieces of data stored in computer memory.
			\item Variables are labels put onto values.
		\end{itemize}
		
		\pause
		
		\begin{lstlisting}
		>>> x = [7, 8]
		>>> y = x
		>>> y.append(11)
		>>> x
		$\pause$[7, 8, 11]
		\end{lstlisting}
		
		Result:
		\begin{itemize}
			\item Assignment does not create a new value.
			\item More than one variable can label the same value.
			% use the words aliasing, reference?
		\end{itemize}
	\end{frame}
	
	\begin{frame}[fragile]
		\frametitle{Slicing}

		\begin{itemize}
			\item Take a section of a string or list:
			\begin{lstlisting}[xleftmargin=\dimexpr-\leftmargini]
			>>> name = "Satyarth"
			>>> nickname = name[0:3]$\pause$
			>>> nickname
			"Sat"
			\end{lstlisting}

			\item \lstinline|[$\colorbox{lightspacebg}{\rule{0pt}{6pt}start}$:$\colorbox{lightspacebg}{end}$]| \\
			means ``from \lstinline|start|, up to (but not including) \lstinline|end|".
			
			\item View the indices as between the elements:
		\end{itemize}
		\begin{tabular}{llllllllllllllllll}
			~ & \color{red}\tt S & ~ & \color{red}\tt a & ~ & \color{red}\tt t & ~ & \color{red}\tt y & ~ & \color{red}\tt a & ~ & \color{red}\tt r & ~ & \color{red}\tt t & ~ & \color{red}\tt h \\
			0 & ~ & 1 & ~ & 2 & ~ & 3 & ~ & 4 & ~ & 5 & ~ & 6 & ~ & 7 & ~ & 8 \\
		\end{tabular}
	\end{frame}
		
	\begin{frame}[fragile]
		\frametitle{Slicing}
		
		\begin{itemize}
			\item \lstinline|start| and \lstinline|end| are optional:
			\begin{columns}[c]
				\column{.45\textwidth}
				\begin{lstlisting}
				>>> "Satyarth"[:3]
				"Sat"
				\end{lstlisting}
				\column{.45\textwidth}
				\begin{lstlisting}
				>>> "Satyarth"[3:]
				"yarth"
				\end{lstlisting}
			\end{columns}
			\note<3>{This is another reason why the end is not included: if it was, ``y" would have appeared in both slices.}
			
			\pause
			
			\item This leads to an idiom for duplicating a list or string:
			\begin{lstlisting}[xleftmargin=\dimexpr-\leftmargini]
			>>> original = [1, 2, 3]
			>>> dup = original[:]
			>>> dup.append(5)
			>>> original, dup
			[1, 2, 3], [1, 2, 3, 5]
			\end{lstlisting}
			
			\pause
			
			\item Third option: step size.
			
			\begin{columns}[c]
				\column{.45\textwidth}
				\begin{lstlisting}
				>>> "Satyarth"[1::2]
				"ayrh"
				\end{lstlisting}
				\column{.45\textwidth}
				\begin{lstlisting}
				>>> "Satyarth"[::-1]
				"htraytaS"
				\end{lstlisting}
			\end{columns}
		\end{itemize}
		
		% x:y, optional (off by one), step
		% why is the end not included? give a good reason
		% - ensures len of result == y-x
		% - x[0:len(x)] shouldn't be a bounds error (would be otherwise)
		%   - i think that last one highlights it: indices are cardinal, not ordinal, numbers
	\end{frame}
	
	\begin{frame}[fragile]
		\frametitle{List comprehensions}
		% if predicate, filtering
	\end{frame}
	
	\begin{frame}[fragile]
		\frametitle{Dictionaries}
		
		\begin{itemize}
			\item Dictionaries allow us to have a bunch of \evenbox{jargonbg}{values} associated with certain \evenbox{jargonbg}{keys}.
			\begin{lstlisting}[xleftmargin=\dimexpr-\leftmargini]
			ages = {"Alice": 42, "Bob": 40}
			\end{lstlisting}
			
			\item One ``looks up" the key and gets the value back.
			\begin{lstlisting}[xleftmargin=\dimexpr-\leftmargini]
			>>> ages["Bob"]
			40\end{lstlisting}
			
			\item Jargon: \lstinline|"Alice"| is said to \evenbox{jargonbg}{map} to \lstinline|42|.
			
			% datatypes that can be used as keys? hashable, immutable
			% strings, floats, integers, tuples (mention?)
			% keys and values do not have to be of one type
			
			\item Adding (or updating) key-value pairs:
			\begin{lstlisting}[xleftmargin=\dimexpr-\leftmargini]
			ages["Charlie"] = 1
			\end{lstlisting}
		\end{itemize}
	\end{frame}
	
	\begin{frame}[fragile]
		\frametitle{Error handling}
		% file access? that would emphasize correctness (race condition)
		
		\begin{itemize}
			\item Looking for a key that is not in the dictionary is an error:
			\begin{lstlisting}[basicstyle=\scriptsize\tt, xleftmargin=\dimexpr-\leftmargini]
			>>> ages[name] = ages[name]+1
			KeyError: "Satyarth"
			\end{lstlisting}
			
			\pause
			
			\vspace{-2mm} \item We can guard against  it:
		\end{itemize}
		
		\vspace{-2mm} \begin{columns}[c]
			\column{.49\textwidth}
			\begin{lstlisting}[basicstyle=\scriptsize\tt]
			if name in ages.keys():
			    ages[name] = ages[name]+1
					
			else:
			    ages[name] = 0
			\end{lstlisting}
			
			\pause
			
			\column{.49\textwidth}
			\begin{lstlisting}[basicstyle=\scriptsize\tt]
			try:
			    ages[name] = ages[name]+1
		
			except KeyError:
			    ages[name] = 0
			\end{lstlisting}
		\end{columns}
		
		\begin{quote}
			It's easier to ask forgiveness than it is to get permission.
		\end{quote} \vspace*{-2mm} \hfill -- Grace Hopper
		
		\pause
		
		Advantages of \lstinline|try|-\lstinline|except|: \\
		\quad $\triangleright$ Simpler
		$\triangleright$ Less error prone
		$\triangleright$ Keeps error handling separate
	\end{frame}
	
	\begin{frame}[fragile]
		\frametitle{}
		
		\begin{columns}[t]
			\column{.45\textwidth}
			\begin{itemize}
				\item Last session we had this code:
				\begin{lstlisting}[xleftmargin=\dimexpr-\leftmargini, basicstyle=\scriptsize\tt]
				def find(list, value):
				    for item in list:
				        if item == value:
				            return True
				
				    return False
				\end{lstlisting}
				
				\item We wanted to change it to return the index where the element was found.
			\end{itemize}
			
			\pause
			
			\column{.45\textwidth}
			\begin{itemize}
				\item You may have come up with something like this:
				\begin{lstlisting}[xleftmargin=\dimexpr-\leftmargini, basicstyle=\scriptsize\tt]
				def find(list, value):
				    index = 0
				
				    for item in list:
				        if item == value:
				            return index
				
				        index += 1
				
				    return False
				\end{lstlisting}
			\end{itemize}
		\end{columns}
	\end{frame}
	
	\begin{frame}[fragile]
		\frametitle{Enumerate}
		
		\begin{itemize}
			\item \lstinline|enumerate($\colorbox{lightspacebg}{list}$)|
			
			\item Pairs each element with its index:
			\begin{lstlisting}[xleftmargin=\dimexpr-\leftmargini]
			>>> enumerate(["Alice", "Bob", "Charlie"])      $\color{blue}*$
			[(0, 'Alice'),
			 (1, 'Bob'),
			 (2, 'Charlie')]
			\end{lstlisting}
			
			\item We can use it to iterate over the list: 
			\begin{lstlisting}[xleftmargin=\dimexpr-\leftmargini]
			def find(list, value):
			    for index, item in enumerate(list):
			        if item == value:
			            return index
			
			    return False
			\end{lstlisting}
			
			\item Each loop iteration receives a pair, \lstinline|(index, item)|, not just one element.
		\end{itemize}
		
		% todo: tuples, unpacking, assignment
	\end{frame}
	
	\begin{frame}[fragile]
		\frametitle{Iterables}
		
		\begin{itemize}
			\item Not everything we iterate over with \lstinline|for| is a list.
			
			\pause
			
			\item \lstinline|enumerate| does not actually give you a list!
			\begin{lstlisting}[xleftmargin=\dimexpr-\leftmargini, basicstyle=\scriptsize\tt]
			>>> enumerate(["Alice", "Bob", "Charlie"])
			<enumerate object at $\text{\tt\color{num} 0x005FA2B0}$>$\pause$
			>>> for item in enumerate(["Alice", "Bob", "Charlie"]):
			...     print(item)
			(0, 'Alice')
			(1, 'Bob')
			(2, 'Charlie')
			\end{lstlisting}
			
			\pause
			
			\item These are special objects which act like lists, called \colorbox{jargonbg}{iterables}.
			
			\item Can be converted to lists with \lstinline|list|:
			\begin{lstlisting}[xleftmargin=\dimexpr-\leftmargini]
			>>> range(10)
			range(0, 10)
			>>> list(range(10))
			[0, 1, 2, 3, 4, 5, 6, 7, 8, 9]
			\end{lstlisting}
			
		\end{itemize}
	\end{frame}
	
	\begin{frame}[fragile]
		\frametitle{Logical operators}
		\begin{itemize}
			\item They operate on booleans.
			\item Allow you to combine expressions together.
			\pause
			\item \lstinline|and|: returns \lstinline|True| if both of the operands are \lstinline|True|, otherwise returns \lstinline|False|.
			\item \lstinline|or|: returns \lstinline|True| if at least one of the operands is \lstinline|True|, otherwise returns \lstinline|False|.
			\item \lstinline|not|: returns the complement of the operand.
		\end{itemize}
	\end{frame}
	
	\begin{frame}[fragile]
		\frametitle{Try it out!}
		
		Print a number if it is divisible by two or five:
		
		\begin{lstlisting}
		for num in range(10):
		    if num % 2 == 0 or num % 5 == 0:
		        print(num)
		\end{lstlisting}
		
		Print a number if it isn't divisible by two or three:
		
		\begin{lstlisting}
		for num in range(20):
		    if not (num % 2 == 0 and num % 3 == 0):
		        print(num)
		\end{lstlisting}
	\end{frame}
	
	\begin{frame}[fragile]
		\frametitle{while}
		\begin{itemize}
			\item Sometimes you want your loop to run while some condition is met, as opposed to a set number of times (\lstinline|for| loops).
			\item Enter the \lstinline|while| loop.
			\pause
			\item Example:
			% Why isn't this listing more left-aligned?
			\begin{lstlisting}[basicstyle=\scriptsize\tt]
			num = 0
			while not num > 0:
			    num = int(input("Enter a positive number: "))
			\end{lstlisting}

		\end{itemize}
	\end{frame}
	
	\begin{frame}[fragile]
		\frametitle{break}
	\end{frame}
	
	\begin{frame}[fragile]
		\frametitle{continue}
		\begin{lstlisting}
		story = """
		        John brought 3 apples and Jenny brought
		        26 pears. Meanwhile, Bastian brought 22
		        guavas!
		        """

		number_of_fruits = 0
		
		for word in story.split():
		    try:
		        number_of_fruits += int(word)
		    except ValueError:
		        continue
		
		    print(number_of_fruits, "lovely fruits!")
		\end{lstlisting}
	\end{frame}
	
	\begin{frame}[fragile]
		\frametitle{Files}
		% use with, damn nice syntax
		
		\begin{lstlisting}
		with open("input.txt") as file:
		    print(file.read())
		\end{lstlisting}
	\end{frame}
\end{document}