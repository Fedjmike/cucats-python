\documentclass[notes]{beamer}

\usepackage{listings}
\usepackage{color}
\usepackage{etoolbox}

\definecolor{orange}{rgb}{1,0.5,0}
\definecolor{green}{rgb}{0,0.5,0}
\definecolor{gray}{rgb}{0.25,0.25,0.25}
\definecolor{mauve}{rgb}{0.58,0,0.82}
\definecolor{red}{rgb}{1.0,0,0}

\colorlet{punct}{green}
\colorlet{delim}{green}
\colorlet{num}{mauve}

\definecolor{codebg}{rgb}{1.0,1.0,0.8}
\definecolor{jargonbg}{rgb}{0.8,1.0,0.8}

\newtoggle{InString}{}% Keep track of if we are within a string
\togglefalse{InString}% Assume not initally in string

\newcommand*{\ColorIfNotInString}[1]{\iftoggle{InString}{#1}{\color{num}#1}}
\newcommand*{\ProcessQuote}[1]{\color{red}#1\iftoggle{InString}{\global\togglefalse{InString}}{\global\toggletrue{InString}}}

\lstdefinelanguage{py}{
	morekeywords={if,elif,else,for,in,def,class,True,False,print},
	literate=
		*{=}{{{\color{punct}{=}}}}{1}
		{|}{{{\color{punct}{|}}}}{1}
		{\^}{{{\color{punct}{\^{}}}}}{1}
		{\{}{{{\color{punct}{\{}}}}{1}
		{\}}{{{\color{punct}{\}}}}}{1}
		{[}{{{\color{punct}{[}}}}{1}
		{]}{{{\color{punct}{]}}}}{1}
		{)}{{{\color{punct}{)}}}}{1}
		{(}{{{\color{punct}{(}}}}{1}
		{<}{{{\color{punct}{<}}}}{1}
		{>}{{{\color{punct}{>}}}}{1}
		{*}{{{\color{punct}{*}}}}{1}
		{?}{{{\color{punct}{?}}}}{1}
		{:}{{{\color{punct}{:}}}}{1}
		{,}{{{\color{punct}{,}}}}{1}
		{\&}{{{\color{punct}{\&}}}}{1}
		{!}{{{\color{punct}{!}}}}{1}
		{-}{{{\color{punct}{-}}}}{1}
		{+}{{{\color{punct}{+}}}}{1}
		{.}{{{\color{punct}{.}}}}{1}
		{\%}{{{\color{punct}{\%}}}}{1}
		{/}{{{\color{punct}{/}}}}{1}
		{\~}{{{$\color{punct}{\sim}$}}}{1}
		{;}{{{\color{punct}{;}}}}{1}
		%{"}{{{\ProcessQuote{"}}}}1% Disable coloring within double quotes
		%{'}{{{\ProcessQuote{'}}}}1% Disable coloring within single quote
		{0}{{{\ColorIfNotInString{0}}}}1
		{1}{{{\ColorIfNotInString{1}}}}1
		{2}{{{\ColorIfNotInString{2}}}}1
		{3}{{{\ColorIfNotInString{3}}}}1
		{4}{{{\ColorIfNotInString{4}}}}1
		{5}{{{\ColorIfNotInString{5}}}}1
		{6}{{{\ColorIfNotInString{6}}}}1
		{7}{{{\ColorIfNotInString{7}}}}1
		{8}{{{\ColorIfNotInString{8}}}}1
		{9}{{{\ColorIfNotInString{9}}}}1,
}

\lstset{
	language=py,
	%
	captionpos=b,                    % sets the caption-position to bottom
	mathescape=true,
	keepspaces=true,
	showspaces=false,
	showstringspaces=false,          % underline spaces within strings only
	showtabs=false,                  % show tabs within strings adding particular underscores
	stepnumber=1,
	tabsize=4,
	title=\lstname,
	morestring=[b]",
	%
	numberstyle=\footnotesize\color{gray},
	rulecolor=\color{black},
	basicstyle=\ttfamily\small,
	commentstyle=\itshape,
	backgroundcolor=\color{codebg},
	keywordstyle=\color{blue},
	commentstyle=\color{orange}\itshape,
	stringstyle=\color{red},
}

\lstdefinestyle{output}{
	basicstyle=\small\ttfamily,
	numbers=none,
	frame=tblr,
	columns=fullflexible,
	backgroundcolor=\color{blue!10},
	linewidth=0.9\linewidth,
	xleftmargin=0.1\linewidth
}

\title{CUCaTS Python Workshop}
\date{Lent 2015}

\begin{document}

\begin{frame}[fragile]
	\maketitle
\end{frame}

\begin{frame}[fragile]
	\begin{quotation}
		Literacy is a bridge from misery to hope. It is a tool for daily life in modern society. It is a bulwark against poverty, and a building block of development, an essential complement to investments in roads, dams, clinics and factories. Literacy is a platform for democratization, and a vehicle for the promotion of cultural and national identity. Especially for girls and women, it is an agent of family health and nutrition. For everyone, everywhere, literacy is, along with education in general, a basic human right.... Literacy is, finally, the road to human progress and the means through which every man, woman and child can realize his or her full potential.
	\end{quotation}
	-- Kofi Annan
\end{frame}

\begin{frame}[fragile]
	\frametitle{Python}
	Python is a language: a set of rules for constructing programs.
	
	You can experiment with these rules using ``IDLE".
	\begin{itemize}
		\item When you open IDLE, you can type things in and press enter to have IDLE read what you wrote.
		\item When you type things in, you should write in Python. IDLE will complain if you don't.
		\item Things you do in IDLE aren't permanent. You can always close IDLE and open it afresh.
		\item As we introduce new ideas today, you can follow along by typing things into IDLE.
	\end{itemize}
\end{frame}

\begin{frame}[fragile]
	\frametitle{Values}
	
	\begin{itemize}
		\item A program processes data.
		\item Data such as numbers, text, images, dates/time, sound, files and so on.
		\item Even complex and abstract things like a road network or the structure of a protein.
		\item A value is a piece of data stored in a computer's memory.
	\end{itemize}
\end{frame}

\begin{frame}[fragile]
	\frametitle{Datatypes}
	
	All values have a datatype. Some examples:
	
	\begin{itemize}
		\item Numbers (integers, floats) \\
		e.g. \colorbox{codebg}{\lstinline|5|} or \colorbox{codebg}{\lstinline|9.6|}
		\note<1>{Integers can hold whole numbers only. ``Floats", or floating-point numbers, are capable of fractional numbers as well. \\}
		
		\pause
		
		\item Booleans \\
		\colorbox{codebg}{\lstinline|True|} and \colorbox{codebg}{\lstinline|False|}
		\note<2>{True/false values are called ``Booleans" (after George Boole). They are used for expression conditions.}
		
		\pause
		
		\item Strings \\
		\colorbox{codebg}{\lstinline|"Hello world!"|}
		\note<3>{Strings can use single quotes or double quotes, but we will stick to double quotes for consistency. \\}
		
		\pause
		
		\item Lists \\
		\colorbox{codebg}{\lstinline|[4, 8, 15, 16, 23, 42]|} \\
		or \colorbox{codebg}{\lstinline|["first", "second", "fourth"]|} \\
		or even \colorbox{codebg}{\lstinline|[[], "some text", 99]|}
		\note<4>{Lists hold a sequence of values. The third example is a list containing firstly an empty list, then a string and then an integer. As you can see, the elements of a list do not have to all be the same datatype. In Python, a list is simply a list, not a list \emph{of} something. \\}
		
		%\pause
		
		%\item Dictionaries
		% bother explaining them this early?
		
		% I'm thinking, don't mention tuples because it will be difficult to explain how they differ from lists
	\end{itemize}
\end{frame}

\begin{frame}[fragile]
	\frametitle{}
	
	Arithmetic:
	
	\begin{lstlisting}
	>>> 5 + 9
	14$\pause$
	>>> 7.5 * 2
	15.0
	\end{lstlisting}
	\note<2>{Explain REPL. Note that the multiplication results in \lstinline|15.0|, a float, because \lstinline|7.5| was a float.}
	
	\pause
	
	Comparisons:
	
	\begin{lstlisting}
	>>> 5 < 6
	False
	>>> 9 == 18 / 2
	True
	\end{lstlisting}
	
	\pause

	\begin{lstlisting}
	>>> "two" + "strings"
	"twostrings" 
	\end{lstlisting}
\end{frame}

\begin{frame}[fragile]
	\frametitle{Try it out}

	\begin{lstlisting}
	>>> "a" + "b"
	>>> "test" + "ing" == "testing"
	>>> 3 == "3"
	>>> "a" < "b"
	\end{lstlisting}
	% Some basic operations on values
	% indexing, appending lists (needed in coming slides)
	% print
	% REPL
\end{frame}

\begin{frame}[fragile]
	\frametitle{Errors}
	
	\begin{itemize}
		\item There are rules governing what you can't do with certain types and values.
		\pause For example:
		\begin{itemize}
			\item Adding an integer to a string.
			\begin{onlyenv}<2>
				\begin{lstlisting}
				>>> 16 + "hello"
				$\textrm{\color{red}TypeError: unsupported operand type(s) for +: `int' and `str'}$
				\end{lstlisting}
			\end{onlyenv}
			
			\pause
			
			\item Indexing into something that is not a sequence.
			\begin{onlyenv}<3>
				\begin{lstlisting}
				>>> 9.0[1]
				$\textrm{\color{red}TypeError: `float' object has no attribute `\_\_getitem\_\_'}$
				\end{lstlisting}
			\end{onlyenv}
			
			\pause
			
			\item Asking for the sixth element of a list that only has four.
			\begin{onlyenv}<4>
				\begin{lstlisting}
				>>> [4, 5, 6, 7][5]
				$\textrm{\color{red}IndexError: list index out of range}$
				\end{lstlisting}
			\end{onlyenv}
			
			\pause
		\end{itemize}
		\item Designed to help you spots problems with your code.
	\end{itemize}
\end{frame}

\begin{frame}[fragile]
	\frametitle{Variables}
	\begin{itemize}
		\item Variables are labels put onto values. They allow us to refer to those values by a name.
		% NOT BOXES BECAUSE REFERENCE SEMANTICS
		\note<1>{Variable names must consist only of letters, digits and underscores. They can't start with a digit or have spaces. They are case sensitive, }
	\end{itemize}
	
	\begin{onlyenv}<1-2>
		\begin{lstlisting}
		>>> x = 5
		>>> y = x+6
		>>> print(y)
		11
		\end{lstlisting}
	\end{onlyenv}
	
	\pause
	
	\begin{itemize} \item They can be reassigned: \end{itemize}
	
	\begin{lstlisting}
	>>> x = 5
	>>> x = 7
	>>> print(x)
	7
	\end{lstlisting}
	
	\pause
	
	\begin{itemize} \item Assignment does not create a new value: \end{itemize}
	% save for later? allows delaying explaining list += list
	
	\begin{lstlisting}
	>>> x = [5, 7]
	>>> y = x
	>>> y += [9, 11]
	>>> print(x)
	[5, 7, 9, 11]
	\end{lstlisting} \note<3>{\lstinline|x| and \lstinline|y| now label the same value. When we add to \lstinline|y|, it is the same list that gets modified.}
\end{frame}

\begin{frame}[fragile]
	\frametitle{Processing lists}
	
	\begin{onlyenv}<1>
	\begin{lstlisting}
	>>> nums = [1, 2, 3, 4, 5]
	>>> squares = [1, 4, 9, 16, 25]
	\end{lstlisting}
	
	~ \\
	~ \\

	\end{onlyenv}
	
	\pause
	
	\begin{onlyenv}<2->
	\begin{lstlisting}
	>>> nums = [1, 2, 3, 4, 5]
	>>> squares = [x*x for x in nums]$\pause$
	>>> print(squares)
	[1, 4, 9, 16, 25]
	\end{lstlisting}
	\end{onlyenv}
	
	\pause
	
	\begin{itemize}
		\item Creates a new list out of another by using each element to create a corresponding new value.
		\item `\lstinline|for|' and `\lstinline|in|' are keywords.
		\note<3->{Keywords are ones that have a special meaning. You cannot use them as variable names. List comprehensions must use `\lstinline|for|' and `\lstinline|in|'}
		\item This is called a \colorbox{jargonbg}{list comprehension}.
	\end{itemize}
\end{frame}


\begin{frame}[fragile]
	\frametitle{for loops}
	
	\begin{lstlisting}
	>>> nums = [1, 2, 3, 4, 5]
	>>> total = 0
	
	>>> for x in nums:
	        total = total + x
	$\pause$
	>>> print(total)
	15
	\end{lstlisting}
	
	% step by step on the side of the running total
	
	% whitespace...
\end{frame}

\begin{frame}[fragile]
	\frametitle{if-else}
	
	Provide two different pieces of code to run depending on some condition.
	
	% explain in the context of for loops, because otherwise the condition is going to be tautological and it won't be clear why providing a branch is useful
	% which is why if comes after for
	
	\begin{lstlisting}
	if :
	\end{lstlisting}
\end{frame}

\begin{frame}[fragile]
	\frametitle{}
	
	% and or gates
	% ==
\end{frame}

\begin{frame}[fragile]
	\frametitle{Functions}
\end{frame}

\begin{frame}[fragile]
	\frametitle{Methods}
	% Use (n.) only, because they're going to encounter that pretty early
\end{frame}

\end{document}